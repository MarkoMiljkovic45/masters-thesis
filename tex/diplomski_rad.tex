\documentclass[diplomskirad, upload]{fer}
% Dodaj opciju upload za generiranje konačne verzije koja se učitava na FERWeb
% Add the option upload to generate the final version which is uploaded to FERWeb


% Ovdje dodati pakete npr. \usepackage{blindtext}


%--- PODACI O RADU / THESIS INFORMATION ----------------------------------------

% Naslov na engleskom jeziku / Title in English
\title{TBD}

% Naslov na hrvatskom jeziku / Title in Croatian
\naslov{TBD}

% Broj rada / Thesis number
\brojrada{1234}

% Autor / Author
\author{Marko Miljković}

% Mentor 
\mentor{doc.~dr.~sc.~Stjepan~Begušić}

% Datum rada na engleskom jeziku / Date in English
\date{February, 2026}

% Datum rada na hrvatskom jeziku / Date in Croatian
\datum{Veljača, 2026.}

%-------------------------------------------------------------------------------


\begin{document}


% Naslovnica se automatski generira / Titlepage is automatically generated
\maketitle


%--- ZADATAK / THESIS ASSIGNMENT -----------------------------------------------

% Zadatak se ubacuje iz vanjske datoteke / Thesis assignment is included from external file
% Upiši ime PDF datoteke preuzete s FERWeb-a / Enter the filename of the PDF downloaded from FERWeb
\zadatak{filename.pdf}


%--- ZAHVALE / ACKNOWLEDGMENT --------------------------------------------------

\begin{zahvale}
  % Ovdje upišite zahvale / Write in the acknowledgment
  TODO Zahvala
\end{zahvale}


% Odovud započinje numeriranje stranica / Page numbering starts from here
\mainmatter


% Sadržaj se automatski generira / Table of contents is automatically generated
\tableofcontents


%--- UVOD / INTRODUCTION -------------------------------------------------------
\chapter{Uvod}
\label{pog:uvod}

TODO Uvod, ovdje ćemo testirati citiranje \cite{johnson1998,tsay2010}.

%-------------------------------------------------------------------------------
\chapter{Metodologija}
\label{pog:metodologija}

Ovo poglavlje pokriva teoretsku osnovu potrebnu za razumijevanje implementiranog modela dubokog učenja te 
samu arhitekturu modela. Prvo ćemo definirati varijable pomoću kojih ćemo kasnije oblikovati naše
podatke. Zatim ćemo proći osnove faktorskih modela i razmotriti kako jednofaktorski model može objasniti
povrate imovina. Na kraju ćemo opisati sam model dubokog učenja, njegovu arhitekturu i funckije cilja
koje se koriste u ovom radu.

\section{Varijable i povrati imovine}
\label{sec:varijable}

Radovi koji se bave financijama uglavnom koriste povrate umijesto samih cijena imovine. Postoje dva glavna
razloga za to. Prvi je to što povrati sažimaju kretanja cijena te ih stavljaju na jednaku i usporedivu skalu.
Drugi se odnosi na povoljnija statistička svojstva povrata u odnosu na cijene. Međutim postoji više načina za
definirati povrate.

Nek je $P_t$ cijena imovine u trenutku $t$. U nastavku ćemo razmotriti definicije povrata koji se koriste
kroz ovaj rad.


\subsection{Jednostavni (artimetički) povrati}
\label{subsec:artimeticki}

Ako imovinu posjedujemo od trenutka $t-1$ do trenutka $t$, dobivamo artimetički \emph{bruto} povrat:
\begin{equation}
  1 + R_t = \frac{P_t}{P_{t-1}} \mspace{48mu} \text{ili} \mspace{48mu} P_t = P_{t-1}(1+R_t)
\label{eq:artimeticki_bruto}
\end{equation}

Odgovarajući artimetički \emph{neto} povrat dan je izrazom:
\begin{equation}
  R_t = \frac{P_t}{P_{t-1}} - 1 = \frac{P_t - P_{t-1}}{P_{t-1}}
\label{eq:artimeticki_neto}
\end{equation}

Ukoliko imovinu držimo kroz $T$ perioda ukupni artimetički \emph{bruto} povrat dobivamo izrazom:
\begin{align}
  1 + R_{total} = \frac{P_T}{P_1} &= \frac{P_T}{P_{T-1}} \times \frac{P_{T-1}}{P_{T-2}} \times ... \times \frac{P_2}{P_1} \\
  &= (1+R_T)(1+R_{T-1})\ ...\ (1+R_1) \\
  &= \prod_{t=1}^{T}(1 + R_t).
\end{align}


\subsection{Kontinuirani (logaritamski) povrati}
\label{subsec:logaritamski}

Prirodni logaritam artimetičkog \emph{bruto} povrata (\ref{eq:artimeticki_bruto}) daje nam kontinuirani
odnosno logaritamski povrat:
\begin{equation}
  r_t = ln(1 + R_t) = ln\frac{P_t}{P_{t-1}} = p_t - p_{t-1},
\label{eq:logartiamski}
\end{equation}
gdje $p_t = ln(P_t)$. 

Jedan od načina za modeliranje ovakvih podataka je pomoću faktorskih modela.


\section{Faktorski modeli}
\label{sec:faktorski}

TODO Objasnjenje faktorskih modela


\section{Jednofaktorski model povrata}
\label{sec:jednofaktorski}

TODO Objašnjenje jednofaktorskog modela


\section{LSTM Model}
\label{sec:lstm}

TODO Objasniti arhitekturu našeg modela, različiti pristupi
TOOD Možda će biti potrebno raščlanit po funkcijama cilja


%-------------------------------------------------------------------------------
\chapter{Podatci}
\label{pog:podatci}

TODO Odakle nam podatci, kako su procesirani, povezat s definiranim varijablama


%-------------------------------------------------------------------------------
\chapter{Rezultati}
\label{pog:rezultati}

TODO Rezultati i rasprava


%--- ZAKLJUČAK / CONCLUSION ----------------------------------------------------
\chapter{Zaključak}
\label{pog:zakljucak}

TODO Zaključak


%--- LITERATURA / REFERENCES ---------------------------------------------------

% Literatura se automatski generira iz zadane .bib datoteke / References are automatically generated from the supplied .bib file
% Upiši ime BibTeX datoteke bez .bib nastavka / Enter the name of the BibTeX file without .bib extension
\bibliography{literatura}



%--- SAŽETAK / ABSTRACT --------------------------------------------------------

% Sažetak na hrvatskom
\begin{sazetak}
  Unesite sažetak na hrvatskom.

\end{sazetak}

\begin{kljucnerijeci}
  prva ključna riječ; druga ključna riječ; treća ključna riječ
\end{kljucnerijeci}


% Abstract in English
\begin{abstract}
  Enter the abstract in English.

\end{abstract}

\begin{keywords}
  the first keyword; the second keyword; the third keyword
\end{keywords}


%--- PRIVITCI / APPENDIX -------------------------------------------------------

% Sva poglavlja koja slijede će biti označena slovom i riječi privitak / All following chapters will be denoted with an appendix and a letter
\backmatter

\chapter{The Code}

TODO Appendix ako bude potrebe


\end{document}
