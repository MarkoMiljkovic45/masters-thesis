\documentclass[diplomskirad, upload]{fer}
% Dodaj opciju upload za generiranje konačne verzije koja se učitava na FERWeb
% Add the option upload to generate the final version which is uploaded to FERWeb


% Ovdje dodati pakete npr. \usepackage{blindtext}
\usepackage{algorithm}
\usepackage{algorithmic}


%--- PODACI O RADU / THESIS INFORMATION ----------------------------------------

% Naslov na engleskom jeziku / Title in English
\title{TBD}

% Naslov na hrvatskom jeziku / Title in Croatian
\naslov{TBD}

% Broj rada / Thesis number
\brojrada{1234}

% Autor / Author
\author{Marko Miljković}

% Mentor 
\mentor{doc.~dr.~sc.~Stjepan~Begušić}

% Datum rada na engleskom jeziku / Date in English
\date{February, 2026}

% Datum rada na hrvatskom jeziku / Date in Croatian
\datum{Veljača, 2026.}

%-------------------------------------------------------------------------------


\begin{document}


% Naslovnica se automatski generira / Titlepage is automatically generated
\maketitle


%--- ZADATAK / THESIS ASSIGNMENT -----------------------------------------------

% Zadatak se ubacuje iz vanjske datoteke / Thesis assignment is included from external file
% Upiši ime PDF datoteke preuzete s FERWeb-a / Enter the filename of the PDF downloaded from FERWeb
\zadatak{zadatak.pdf}


%--- ZAHVALE / ACKNOWLEDGMENT --------------------------------------------------

\begin{zahvale}
  % Ovdje upišite zahvale / Write in the acknowledgment
  TODO Zahvala
\end{zahvale}


% Odovud započinje numeriranje stranica / Page numbering starts from here
\mainmatter


% Sadržaj se automatski generira / Table of contents is automatically generated
\tableofcontents


%--- UVOD / INTRODUCTION -------------------------------------------------------
\chapter{Uvod}
\label{pog:uvod}

TODO Uvod.

%-------------------------------------------------------------------------------
\chapter{Metodologija}
\label{pog:metodologija}

Ovo poglavlje pokriva teoretsku osnovu potrebnu za razumijevanje implementiranog modela dubokog učenja te 
samu arhitekturu modela. Prvo ćemo definirati varijable pomoću kojih ćemo kasnije oblikovati naše
podatke. Zatim ćemo proći osnove faktorskih modela i razmotriti kako jednofaktorski model može objasniti
povrate imovina. Na kraju ćemo opisati sam model dubokog učenja, njegovu arhitekturu i funckije cilja
koje se koriste u ovom radu. !TODO Revidirat nakon ostatka poglavlja!

\section{Povrati imovine}
\label{sec:povrati}

Vrijednosni papiri predstavljaju financijske instrumente koji potvrđuju određena imovinska ili
druga prava njihova vlasnika, poput prava na udio u vlasništvu poduzeća (dionice) ili prava na
povrat uloženih sredstava uz kamatu (obveznice). Njima se trguje na organiziranim tržištima kapitala,
poput burzi, gdje se kupnja i prodaja odvijaju putem ovlaštenih posrednika, a cijene se formiraju na
temelju ponude i potražnje. Tržišna cijena vrijednosnog papira u svakom trenutku odražava ravnotežu
između kupaca i prodavatelja, pri čemu se svaka realizirana transakcija bilježi kao nova referentna cijena.
Tržišni indeksi predstavljaju promjene vrijednosti grupe dionica ili drugih financijskih
instrumenata. Služe kao mjerilo performansi tržišta ili određenog segmenta tržišta \cite{anafin02}.

Kako bi se omogućila analiza kretanja cijena kroz vrijeme, kontinuirani tok transakcijskih podataka
uzorkuje se u diskretnim vremenskim intervalima (npr. minute, sati, dani), čime se dobivaju vremenski
nizovi cijena koji služe kao temelj za statističku analizu i modeliranje. Slika \ref{fig:crobex-price}
prikazuje kretanje dnevnih cijena CROBEX indeksa u zadnjih 5 godina.

\begin{figure}[htb]
\centering
\includegraphics[width=15cm]{img/crobex_price_plot.png}
\caption{Kretanje cijene CROBEX indeksa}
\label{fig:crobex-price}
\end{figure}

U ovom radu koristit ćemo povrate umijesto samih cijena imovina. Postoje dva glavna razloga za to.
Prvi je to što povrati sažimaju kretanja cijena te ih stavljaju na jednaku i usporedivu skalu.
Drugi se odnosi na povoljnija statistička svojstva povrata u odnosu na cijene. Postoji više načina za
definirati povrate \cite{campbell1997}.


\subsubsection{Jednostavni (artimetički) povrati}
\label{subsubsec:artimeticki}

Nek je $P_t$ cijena imovine u trenutku $t$. Ako imovinu posjedujemo od trenutka $t-1$ do trenutka $t$,
ostvareni artimetički povrat dobivamo izrazom:
\begin{align}
  1 + R_t &= \frac{P_t}{P_{t-1}} \label{eq:artimeticki_gross} \\[12pt]
  R_t &= \frac{P_t}{P_{t-1}} - 1 = \frac{P_t - P_{t-1}}{P_{t-1}} \label{eq:artimeticki}
\end{align}

Ukoliko imovinu držimo kroz $T$ perioda ukupni artimetički povrat dobivamo ukamaćivanjem artimetičkih
povrata u vremenu \cite{tsay2010}:
\begin{equation} \label{eq:artimeticki-ukamacivanje}
\begin{split}
  1 + R_{total} = \frac{P_T}{P_1} &= \frac{P_T}{P_{T-1}} \times \frac{P_{T-1}}{P_{T-2}} \times ... \times \frac{P_2}{P_1} \\
  &= (1+R_T)(1+R_{T-1})\ ...\ (1+R_1) \\
  &= \prod_{t=1}^{T}(1 + R_t).
\end{split}
\end{equation}


\subsubsection{Kontinuirani (logaritamski) povrati}
\label{subsubsec:logaritamski}

Prirodni logaritam artimetičkog povrata (\ref{eq:artimeticki_gross}) je logaritamski povrat:
\begin{equation}
  r_t = \ln(1 + R_t) = \ln\frac{P_t}{P_{t-1}} = p_t - p_{t-1},
\label{eq:logartiamski}
\end{equation}
gdje $p_t = \ln(P_t)$.

Razmotrimo zatim ukamaćivanje logaritamskih povrata u vremenu:
\begin{equation} \label{eq:logaritamski-ukamacivanje}
\begin{split}
  r_{total} &= \ln(1 + R_{total}) = \ln[(1+R_T)(1+R_{T-1})\ ...\ (1+R_1)] \\
  &= \ln(1+R_T) + \ln(1+R_{T-1}) +\ ...\ + \ln(1+R_1) \\
  &= r_T + r_{T-1} + \ ... \ + r_1 \\
  &= \sum_{t=1}^{T}r_t
\end{split}
\end{equation}
Ukamaćivanje logaritamski povrata može se ostvariti zbrajanjem pojedinačnih logaritamskih
povrata. To svojstvo zovemo \emph{aditivnost u vremenu}. Logaritmski povrati također
posjeduju poželjna statistička svojstva \cite{tsay2010}. Za male magnitude vrijedi
$r_t \approx R_t$. Ova aproksimacija je korisna kada se razmatraju kratki vremenski
intervali \cite{anafin02}. Slika \ref{fig:crobex-returns} prikazuje dnevne artimetičke
i logaritamske povrate CROBEX indeksa u zadnih 5 godina. Sa slike vidimo da nema značajne
razlike između artimetičkih i logaritamskih povrata jer su povrati izračunati na kratkom
odnosno dnevnom vremenskom intervalu.

\begin{figure}[htb]
\centering
\includegraphics[width=15cm]{img/crobex_returns_plot.png}
\caption{Artimetički i logaritamski povrati CROBEX indeksa}
\label{fig:crobex-returns}
\end{figure}


\subsubsection{Povrat portfelja}
\label{subsubsec:portfelj}

Artimetički povrat portfelja koji se sastoji od $N$ vrijednosnica je otežana artimetička sredina
\engl{ weighted average} artimetičkih povrata vrijednosnica, gdje je težina \engl{ weight} pojedine
vrijednosnice njezin udio u vrijednosti portfelja \cite{tsay2010}.

Nek je $p$ portfelj kojem je $w_i$ ponder vrijednosnice $i$. Tada je artimetički povrat portfelja $p$ u
trenutku $t$ dan izrazom:
\begin{equation}
  R_{p,t} = \sum_{i=1}^{N}w_iR_{i,t},
\label{eq:portfelj}
\end{equation}
gdje je $R_{i,t}$ artimetički povrat vrijednosnice $i$ u trenutku $t$. Ovo svojstvo nazivamo \emph{aditivnost
u prostoru vrijednosnica}. Logaritamski povrati ne posjeduju to svojstvo.


\subsubsection{Višak povrata}
\label{subsubsec:excess}

Višak povrata \engl{ excess return} predstavlja razliku između ostvarenog povrata određene imovine
i referentnog, bezrizičnog povrata, te mjeri dodatnu kompenzaciju koju investitor ostvaruje za
preuzimanje rizika. Kao referentni povrat najčešće se koriste prinosi na državne obveznice visoke
kreditne kvalitete i kratkog dospijeća, jer se smatra da nose zanemariv rizik \cite{tsay2010}.

\begin{equation}
  R_t^{excess} = R_t - R_f
\label{eq:excess}
\end{equation}


\section{Statistički modeli povrata}
\label{sec:statisticki}

Prije nego razmotrimo konkretne statističke modele povrata, potrebno je ukratko opisati osnovne
distribucije na kojima će se temeljiti daljnja analiza te načine estimacije njihovih parametara.

\subsubsection{Normalna distribucija}
\label{subsubsec:normal}

Normalna distribucija jedan je od najčešće korištenih probabilističkih modela u statistici i
financijama zbog svoje matematičke jednostavnosti i dobrih teorijskih svojstava.
Ako slučajna varijabla $X$ slijedi normalnu distribuciju sa sredinom $\mu$ i varijancom $\sigma^2$,
tada njezina funkcija gustoće ima oblik \cite{johnson1994}:
\begin{equation}
  f(x) = \frac{1}{\sqrt{2\pi\sigma^2}} \exp\left(-\frac{(x-\mu)^2}{2\sigma^2}\right).
\end{equation}

Za osmotreni uzorak $\{x_1, x_2, \dots, x_n\}$, procjena sredine dana je aritmetičkom sredinom
\begin{equation}
  \hat{\mu} = \frac{1}{n}\sum_{i=1}^{n} x_i,
\end{equation}
dok se varijanca procjenjuje izrazom:
\begin{equation}
  \hat{\sigma}^2 = \frac{1}{n}\sum_{i=1}^{n} (x_i - \hat{\mu})^2.
\end{equation}


\subsubsection{Lognormalna distribucija}
\label{subsubsec:lognormal}

Lognormalna distribucija koristi se za modeliranje slučajnih varijabli čiji je logaritam
normalno distribuiran. Ako postoji broj $a$ takav da $Y=\ln(X - a)$ prati normalnu
distribuciju, slučajna varijabla $X$ tada slijedi lognormalnu distribuciju. Kako bi to vrijedilo,
vjerojatnost da slučajna varijabla $X$ poprimi vrijednost manju od $a$ mora biti jednaka nuli.
Ako vrijedi $Y \sim \mathcal{N}(\mu, \sigma^2)$, tada $X$ ima gustoću:
\begin{equation}
  f(x) = \frac{1}{(x-a)\sigma\sqrt{2\pi}} \exp\left[-\frac{(\ln(x - a) - \mu)^2}{2\sigma^2}\right]
  , \quad x > a.
\label{eq:lognormalna-gustoca}
\end{equation}

Sredina i varijanca lognormalne distribucije dane su izrazima
\begin{equation}
  E[X] = e^{\mu + \frac{\sigma^2}{2}}, \quad
  \mathrm{Var}[X] = (e^{\sigma^2}-1)e^{2\mu+\sigma^2}.
\end{equation}

Procjena parametara provodi se logaritamskom transformacijom uzorka, nakon čega se primjenjuju
standardne procjene za normalnu distribuciju \cite{johnson1994}.


\subsubsection{Studentova t distribucija}
\label{subsubsec:studentt}

Studentova t distribucija predstavlja generalizaciju normalne distribucije s dodatnim parametrom
broja stupnjeva slobode $\nu$, koji kontrolira debljinu repova. Za manje vrijednosti $\nu$
distribucija ima izraženije repove, čime omogućuje robusnije modeliranje ekstremnih vrijednosti.

Gustoća Studentove t distribucije sa sredinom $\mu$, varijancom $\sigma^2$ i $\nu$ stupnjeva
slobode ima oblik:
\begin{equation}
f(x) = \frac{\Gamma\left(\frac{\nu+1}{2}\right)}{\Gamma\left(\frac{\nu}{2}\right)\sqrt{\nu\pi\sigma^2}}
\left(1 + \frac{(x-\mu)^2}{\nu\sigma^2}\right)^{-\frac{\nu+1}{2}}.
\end{equation}

Zbog složenog oblika funckije gustoće, ne postoji izravan način za izračun njenih prametara.
Za procjenu parametara t distribucije koriste se metode procjene najveće izglednosti \engl{ maximum
likelihood estimation}.


\subsubsection{Statistička svojstva povrata}

U statističkoj analizi financijskih vremenskih nizova često se polazi od pretpostavke da su aritmetički
povrati nezavisne i jednako distribuirane slučajne varijable s normalnom distribucijom,
konstantnom sredinom i varijancom, čime se značajno pojednostavljuje teorijska obrada i izvođenje
analitičkih rezultata. Međutim, takva pretpostavka suočava se s nizom ograničenja: aritmetički povrati
imaju donju granicu od $-1$, dok normalna distribucija nema ograničenja po realnoj osi,
višeperiodni povrati ne zadržavaju normalnu distribuciju zbog multiplikativne prirode
\ref{eq:artimeticki-ukamacivanje}, a empirijski podaci često pokazuju odstupanja od normalnosti.

Alternativno možemo pretpostaviti da su logaritamski povrati nezavisno i identično normalno
distribuirani sa sredinom $\mu$ i varijancom $\sigma^2$. S tom pretpostavkom impliciramo da
su aritmetički povrati nezavisno i identično lognormalno distribuirani.

Ovaj pristup ima povoljnija matematička svojstva jer su zbrojevi logaritamskih povrata kroz više
razdoblja također normalno distribuirani \ref{eq:logaritamski-ukamacivanje}, a pritom se prirodno
zadovoljava donja granica aritmetičkog povrata (\ref{eq:lognormalna-gustoca}). Unatoč tim prednostima,
ni pretpostavka lognormalnosti u potpunosti ne opisuje realna tržišna kretanja, budući da stvarni
financijski povrati često pokazuju deblje repove distribucije i veću učestalost ekstremnih vrijednosti nego
što to predviđa normalna distribucija.

Slika \ref{fig:crobex-returns-pdf} prikazuje funkcije
gustoće vjerojatnosti dnevnih artimetičkih i logaritamskih povrata CROBEX indeksa u zadnjih
5 godina. Također je prikazana gustoća normalne distribucije sa sredinom i varijancom procjenjenom
iz uzoraka povrata.

\begin{figure}[htb]
\centering
\includegraphics[width=15cm]{img/crobex_pdfs.png}
\caption{Distribucija dnevnih artimetičkih i logaritamskih povrata CROBEX indeksa}
\label{fig:crobex-returns-pdf}
\end{figure}


\section{Faktorski modeli}
\label{sec:faktorski}

TODO Objasnjenje faktorskih modela


\subsection{Jednofaktorski model povrata}
\label{subsec:jednofaktorski}

TODO Objašnjenje jednofaktorskog modela


\subsubsection{Procjena parametara jednofaktorskog modela}
\label{subsubsec:jednofaktorski-estimacija}

MLE estimacija parametara pomoću OLS


\section{LSTM Model}
\label{sec:lstm}

TODO Objasniti arhitekturu našeg modela, različiti pristupi
TOOD Možda će biti potrebno raščlanit po funkcijama cilja

\subsubsection{Multivariatni povrati}
\label{subsubsec:multivariatni}

Nek je $\mathbf{r}_t = (r_{1,t}, \dots, r_{N,t})^T$ vektor log povrata $N$ vrijednosnica u trenutku $t$.
U ovom radu zanimat će nas zajednička distribucija ${\mathbf{r}_t}_{t=1}^{T}$, a posebnu pažnju obratit
ćemo na funkciju uvjetne distribucije
$F(\mathbf{r}_T, \dots,\mathbf{r}_{T-k}|\mathbf{r}_{T-k-1}, \dots, \mathbf{r}_1, \boldsymbol{\theta})$


%-------------------------------------------------------------------------------
\chapter{Podatci}
\label{pog:podatci}

TODO Odakle nam podatci, kako su procesirani, povezat s definiranim varijablama


%-------------------------------------------------------------------------------
\chapter{Rezultati}
\label{pog:rezultati}

TODO Rezultati i rasprava


%--- ZAKLJUČAK / CONCLUSION ----------------------------------------------------
\chapter{Zaključak}
\label{pog:zakljucak}

TODO Zaključak


%--- LITERATURA / REFERENCES ---------------------------------------------------

% Literatura se automatski generira iz zadane .bib datoteke / References are automatically generated from the supplied .bib file
% Upiši ime BibTeX datoteke bez .bib nastavka / Enter the name of the BibTeX file without .bib extension
\bibliography{literatura}



%--- SAŽETAK / ABSTRACT --------------------------------------------------------

% Sažetak na hrvatskom
\begin{sazetak}
  Unesite sažetak na hrvatskom.

\end{sazetak}

\begin{kljucnerijeci}
  prva ključna riječ; druga ključna riječ; treća ključna riječ
\end{kljucnerijeci}


% Abstract in English
\begin{abstract}
  Enter the abstract in English.

\end{abstract}

\begin{keywords}
  the first keyword; the second keyword; the third keyword
\end{keywords}


%--- PRIVITCI / APPENDIX -------------------------------------------------------

% Sva poglavlja koja slijede će biti označena slovom i riječi privitak / All following chapters will be denoted with an appendix and a letter
\backmatter

\chapter{The Code}

TODO Appendix ako bude potrebe


\end{document}
