\documentclass[diplomskirad, upload]{fer}
% Dodaj opciju upload za generiranje konačne verzije koja se učitava na FERWeb
% Add the option upload to generate the final version which is uploaded to FERWeb


% Ovdje dodati pakete npr. \usepackage{blindtext}
\usepackage{algorithm}
\usepackage{algorithmic}


%--- PODACI O RADU / THESIS INFORMATION ----------------------------------------

% Naslov na engleskom jeziku / Title in English
\title{TBD}

% Naslov na hrvatskom jeziku / Title in Croatian
\naslov{TBD}

% Broj rada / Thesis number
\brojrada{1234}

% Autor / Author
\author{Marko Miljković}

% Mentor 
\mentor{doc.~dr.~sc.~Stjepan~Begušić}

% Datum rada na engleskom jeziku / Date in English
\date{February, 2026}

% Datum rada na hrvatskom jeziku / Date in Croatian
\datum{Veljača, 2026.}

%-------------------------------------------------------------------------------


\begin{document}


% Naslovnica se automatski generira / Titlepage is automatically generated
\maketitle


%--- ZADATAK / THESIS ASSIGNMENT -----------------------------------------------

% Zadatak se ubacuje iz vanjske datoteke / Thesis assignment is included from external file
% Upiši ime PDF datoteke preuzete s FERWeb-a / Enter the filename of the PDF downloaded from FERWeb
\zadatak{zadatak.pdf}


%--- ZAHVALE / ACKNOWLEDGMENT --------------------------------------------------

\begin{zahvale}
  % Ovdje upišite zahvale / Write in the acknowledgment
  TODO Zahvala
\end{zahvale}


% Odovud započinje numeriranje stranica / Page numbering starts from here
\mainmatter


% Sadržaj se automatski generira / Table of contents is automatically generated
\tableofcontents


%--- UVOD / INTRODUCTION -------------------------------------------------------
\chapter{Uvod}
\label{pog:uvod}

TODO Uvod.

%-------------------------------------------------------------------------------
\chapter{Metodologija}
\label{pog:metodologija}

Ovo poglavlje pokriva teoretsku osnovu potrebnu za razumijevanje implementiranog modela dubokog učenja te 
samu arhitekturu modela. Prvo ćemo definirati varijable pomoću kojih ćemo kasnije oblikovati naše
podatke. Zatim ćemo proći osnove faktorskih modela i razmotriti kako jednofaktorski model može objasniti
povrate imovina. Na kraju ćemo opisati sam model dubokog učenja, njegovu arhitekturu i funckije cilja
koje se koriste u ovom radu. !TODO Revidirat nakon ostatka poglavlja!

\section{Povrati imovine}
\label{sec:povrati}

Vrijednosni papiri predstavljaju financijske instrumente koji potvrđuju određena imovinska ili
druga prava njihova vlasnika, poput prava na udio u vlasništvu poduzeća (dionice) ili prava na
povrat uloženih sredstava uz kamatu (obveznice). Njima se trguje na organiziranim tržištima kapitala,
poput burzi, gdje se kupnja i prodaja odvijaju putem ovlaštenih posrednika, a cijene se formiraju na
temelju ponude i potražnje. Tržišna cijena vrijednosnog papira u svakom trenutku odražava ravnotežu
između kupaca i prodavatelja, pri čemu se svaka realizirana transakcija bilježi kao nova referentna cijena.
Tržišni indeksi predstavljaju promjene vrijednosti grupe dionica ili drugih financijskih
instrumenata. Služe kao mjerilo performansi tržišta ili određenog segmenta tržišta \cite{anafin02}.

Kako bi se omogućila analiza kretanja cijena kroz vrijeme, kontinuirani tok transakcijskih podataka
uzorkuje se u diskretnim vremenskim intervalima (npr. minute, sati, dani), čime se dobivaju vremenski
nizovi cijena koji služe kao temelj za statističku analizu i modeliranje. Slika \ref{fig:crobex-price}
prikazuje kretanje dnevnih cijena CROBEX indeksa u zadnjih 5 godina.

\begin{figure}[htb]
\centering
\includegraphics[width=15cm]{img/crobex_price_plot.png}
\caption{Kretanje cijene CROBEX indeksa}
\label{fig:crobex-price}
\end{figure}

U ovom radu koristit ćemo povrate umijesto samih cijena imovina. Postoje dva glavna razloga za to.
Prvi je to što povrati sažimaju kretanja cijena te ih stavljaju na jednaku i usporedivu skalu.
Drugi se odnosi na povoljnija statistička svojstva povrata u odnosu na cijene. Postoji više načina za
definirati povrate \cite{campbell1997}.


\subsubsection{Jednostavni (artimetički) povrati}
\label{subsubsec:artimeticki}

Nek je $P_t$ cijena imovine u trenutku $t$. Ako imovinu posjedujemo od trenutka $t-1$ do trenutka $t$,
ostvareni artimetički povrat dobivamo izrazom:
\begin{align}
  1 + R_t &= \frac{P_t}{P_{t-1}} \label{eq:artimeticki_gross} \\[12pt]
  R_t &= \frac{P_t}{P_{t-1}} - 1 = \frac{P_t - P_{t-1}}{P_{t-1}} \label{eq:artimeticki}
\end{align}

Ukoliko imovinu držimo kroz $T$ perioda ukupni artimetički povrat dobivamo ukamaćivanjem artimetičkih
povrata u vremenu \cite{tsay2010}:
\begin{equation} \label{eq:artimeticki-ukamacivanje}
\begin{split}
  1 + R_{total} = \frac{P_T}{P_1} &= \frac{P_T}{P_{T-1}} \times \frac{P_{T-1}}{P_{T-2}} \times ... \times \frac{P_2}{P_1} \\
  &= (1+R_T)(1+R_{T-1})\ ...\ (1+R_1) \\
  &= \prod_{t=1}^{T}(1 + R_t).
\end{split}
\end{equation}


\subsubsection{Kontinuirani (logaritamski) povrati}
\label{subsubsec:logaritamski}

Prirodni logaritam artimetičkog povrata (\ref{eq:artimeticki_gross}) je logaritamski povrat:
\begin{equation}
  r_t = \ln(1 + R_t) = \ln\frac{P_t}{P_{t-1}} = p_t - p_{t-1},
\label{eq:logartiamski}
\end{equation}
gdje $p_t = \ln(P_t)$.

Razmotrimo zatim ukamaćivanje logaritamskih povrata u vremenu:
\begin{equation} \label{eq:logaritamski-ukamacivanje}
\begin{split}
  r_{total} &= \ln(1 + R_{total}) = \ln[(1+R_T)(1+R_{T-1})\ ...\ (1+R_1)] \\
  &= \ln(1+R_T) + \ln(1+R_{T-1}) +\ ...\ + \ln(1+R_1) \\
  &= r_T + r_{T-1} + \ ... \ + r_1 \\
  &= \sum_{t=1}^{T}r_t
\end{split}
\end{equation}
Ukamaćivanje logaritamski povrata može se ostvariti zbrajanjem pojedinačnih logaritamskih
povrata. To svojstvo zovemo \emph{aditivnost u vremenu}. Logaritmski povrati također
posjeduju poželjna statistička svojstva \cite{tsay2010}. Za male magnitude vrijedi
$r_t \approx R_t$. Ova aproksimacija je korisna kada se razmatraju kratki vremenski
intervali \cite{anafin02}. Slika \ref{fig:crobex-returns} prikazuje dnevne artimetičke
i logaritamske povrate CROBEX indeksa u zadnih 5 godina. Sa slike vidimo da nema značajne
razlike između artimetičkih i logaritamskih povrata jer su povrati izračunati na kratkom
odnosno dnevnom vremenskom intervalu.

\begin{figure}[htb]
\centering
\includegraphics[width=15cm]{img/crobex_returns_plot.png}
\caption{Artimetički i logaritamski povrati CROBEX indeksa}
\label{fig:crobex-returns}
\end{figure}


\subsubsection{Povrat portfelja}
\label{subsubsec:portfelj}

Artimetički povrat portfelja koji se sastoji od $N$ vrijednosnica je otežana artimetička sredina
\engl{ weighted average} artimetičkih povrata vrijednosnica, gdje je težina \engl{ weight} pojedine
vrijednosnice njezin udio u vrijednosti portfelja \cite{tsay2010}.

Nek je $p$ portfelj kojem je $w_i$ ponder vrijednosnice $i$. Tada je artimetički povrat portfelja $p$ u
trenutku $t$ dan izrazom:
\begin{equation}
  R_{pt} = \sum_{i=1}^{N}w_iR_{it},
\label{eq:portfelj}
\end{equation}
gdje je $R_{it}$ artimetički povrat vrijednosnice $i$ u trenutku $t$. Ovo svojstvo nazivamo \emph{aditivnost
u prostoru vrijednosnica}. Logaritamski povrati ne posjeduju to svojstvo.


\subsubsection{Povrati iznad bezritične kamatne stope}
\label{subsubsec:excess}

Povrati iznad bezrizične kamatne stope \engl{ excess return} predstavlja razliku između ostvarenog
povrata određene imovine i referentnog, bezrizičnog povrata, te mjeri dodatnu kompenzaciju koju
investitor ostvaruje za preuzimanje rizika. Kao referentni povrat najčešće se koriste prinosi
na državne obveznice visoke kreditne kvalitete i kratkog dospijeća, jer se smatra da nose
zanemariv rizik \cite{tsay2010}.

\begin{equation}
  R_t^{excess} = R_t - R_f
\label{eq:excess}
\end{equation}


\section{Statistički modeli povrata}
\label{sec:statisticki}

Prije nego razmotrimo konkretne statističke modele povrata, potrebno je ukratko opisati osnovne
distribucije na kojima će se temeljiti daljnja analiza te načine estimacije njihovih parametara.

\subsubsection{Normalna distribucija}
\label{subsubsec:normal}

Normalna distribucija jedan je od najčešće korištenih probabilističkih modela u statistici i
financijama zbog svoje matematičke jednostavnosti i dobrih teorijskih svojstava.
Ako slučajna varijabla $X$ slijedi normalnu distribuciju sa sredinom $\mu$ i varijancom $\sigma^2$,
tada njezina funkcija gustoće ima oblik \cite{johnson1994vol1}:
\begin{equation}
  f(x) = \frac{1}{\sqrt{2\pi\sigma^2}} \exp\left(-\frac{(x-\mu)^2}{2\sigma^2}\right).
\end{equation}

Za osmotreni uzorak $\{x_1, x_2, \dots, x_n\}$, procjena sredine dana je aritmetičkom sredinom
\begin{equation}
  \widehat{\mu} = \frac{1}{n}\sum_{i=1}^{n} x_i,
\end{equation}
dok se varijanca procjenjuje izrazom:
\begin{equation}
  \widehat{\sigma^2} = \frac{1}{n}\sum_{i=1}^{n} (x_i - \widehat{\mu})^2.
\end{equation}


\subsubsection{Lognormalna distribucija}
\label{subsubsec:lognormal}

Lognormalna distribucija koristi se za modeliranje slučajnih varijabli čiji je logaritam
normalno distribuiran. Ako postoji broj $a$ takav da $Y=\ln(X - a)$ prati normalnu
distribuciju, slučajna varijabla $X$ tada slijedi lognormalnu distribuciju. Kako bi to vrijedilo,
vjerojatnost da slučajna varijabla $X$ poprimi vrijednost manju od $a$ mora biti jednaka nuli.
Ako vrijedi $Y \sim \mathcal{N}(\mu, \sigma^2)$, tada $X$ ima gustoću:
\begin{equation}
  f(x) = \frac{1}{(x-a)\sigma\sqrt{2\pi}} \exp\left[-\frac{(\ln(x - a) - \mu)^2}{2\sigma^2}\right]
  , \quad x > a.
\label{eq:lognormalna-gustoca}
\end{equation}

Sredina i varijanca lognormalne distribucije dane su izrazima
\begin{equation}
  E[X] = e^{\mu + \frac{\sigma^2}{2}}, \quad
  \mathrm{Var}[X] = (e^{\sigma^2}-1)e^{2\mu+\sigma^2}.
\end{equation}

Procjena parametara provodi se logaritamskom transformacijom uzorka, nakon čega se primjenjuju
standardne procjene za normalnu distribuciju \cite{johnson1994vol1}.


\subsubsection{Studentova t distribucija}
\label{subsubsec:studentt}

Studentova t distribucija predstavlja generalizaciju normalne distribucije s dodatnim parametrom
broja stupnjeva slobode $\nu$, koji kontrolira debljinu repova. Za manje vrijednosti $\nu$
distribucija ima izraženije repove, čime omogućuje robusnije modeliranje ekstremnih vrijednosti.

Gustoća Studentove t distribucije sa sredinom $\mu$, varijancom $\sigma^2$ i $\nu$ stupnjeva
slobode ima oblik:
\begin{equation}
f(x) = \frac{\Gamma\left(\frac{\nu+1}{2}\right)}{\Gamma\left(\frac{\nu}{2}\right)\sqrt{\nu\pi\sigma^2}}
\left(1 + \frac{(x-\mu)^2}{\nu\sigma^2}\right)^{-\frac{\nu+1}{2}}.
\end{equation}

Zbog složenog oblika funckije gustoće, ne postoji izravan način za izračun njenih prametara.
Za procjenu parametara t distribucije koriste se metode procjene najveće izglednosti \engl{ maximum
likelihood estimation} \cite{johnson1994vol2}.


\subsubsection{Statistička svojstva povrata}

U statističkoj analizi financijskih vremenskih nizova često se polazi od pretpostavke da su aritmetički
povrati nezavisne i jednako distribuirane slučajne varijable s normalnom distribucijom,
konstantnom sredinom i varijancom, čime se značajno pojednostavljuje teorijska obrada i izvođenje
analitičkih rezultata. Međutim, takva pretpostavka suočava se s nizom ograničenja: aritmetički povrati
imaju donju granicu od $-1$, dok normalna distribucija nema ograničenja po realnoj osi,
višeperiodni povrati ne zadržavaju normalnu distribuciju zbog multiplikativne prirode
(\ref{eq:artimeticki-ukamacivanje}), a empirijski podaci često pokazuju odstupanja od normalnosti.

Alternativno možemo pretpostaviti da su logaritamski povrati nezavisno i identično normalno
distribuirani sa sredinom $\mu$ i varijancom $\sigma^2$. S tom pretpostavkom impliciramo da
su aritmetički povrati nezavisno i identično lognormalno distribuirani.
Ovaj pristup ima povoljnija matematička svojstva jer su zbrojevi logaritamskih povrata kroz više
razdoblja također normalno distribuirani (\ref{eq:logaritamski-ukamacivanje}), a pritom se prirodno
zadovoljava donja granica aritmetičkog povrata (\ref{eq:lognormalna-gustoca}). Unatoč tim prednostima,
ni pretpostavka lognormalnosti u potpunosti ne opisuje realna tržišna kretanja, budući da stvarni
financijski povrati često pokazuju deblje repove distribucije i veću učestalost ekstremnih vrijednosti nego
što to predviđa normalna distribucija \cite{tsay2010}.

Kako bismo bolje obuhvatili ta empirijska svojstva, možemo koristiti studentovu $t$-distribuciju.
Slika \ref{fig:crobex-returns-pdf} prikazuje distribuciju gustoće vjerojatnosti dnevnih
artimetičkih i logaritamskih povrata CROBEX indeksa u zadnjih 5 godina.
Također je prikazana distribucija gustoće vjerojatnosti normalne distribucije i
studentove $t$-distribucije s pet stupnjeva slobode \footnote{Biramo pet stupnjeva slobode kako bi
imali konačna prva četiri momenta distribucije. Za više informacija pogledaj
\cite[Chapter 28.]{johnson1994vol2}}. Parametri $\mu$ i $\sigma^2$ za normalnu i $t_5$-distribuciju
procjenjeni su iz artimetičkih povrata za lijevi graf i logaritamskih povrata za desni graf.
Sa slike vidimo kako $t_5$-distribucija puno vijernije modelira distribuciju povrata. Također vidimo
da su, zbog male magnitude dnevnih povrata, artimetički i logaritamski povrati gotovo identično
distribuirani.

\begin{figure}[htb]
\centering
\includegraphics[width=15cm]{img/crobex_pdfs.png}
\caption{Distribucija dnevnih artimetičkih i logaritamskih povrata CROBEX indeksa}
\label{fig:crobex-returns-pdf}
\end{figure}


\subsubsection{Slučajni vektori}
\label{subsubsec:slucajni-vektori}

Nek je $\mathbf{X} = (X_1, \dots, X_p)$ slučajni vektor $p$ slučajnih varijabli.
Tada su vektor sredina i kovarijacijska matrica dani izrazima:
\begin{align}
  E(\mathbf{X}) &= \boldsymbol{\mu}_X = \left[E(X_1), \dots, E(X_p)\right]^\intercal \\
  \mathrm{Cov}(\mathbf{X}) &= \mathbf{\Sigma}_X =
  E\left[\left(\mathbf{X} - \boldsymbol{\mu}_X\right)
  \left(\mathbf{X} - \boldsymbol{\mu}_X\right)^\intercal\right]\text{,}
\end{align}
uz uvijet da dana očekivanj postoje. Neka su $\left\{\mathbf{x_1}, \dots, \mathbf{x_T}\right\}$
realizacije slučajnog vektora $\mathbf{X}$. Tada su uzoračka sredina i kovarijacijska matrica dane
izrazima \cite{tsay2010}:
\begin{equation}
  \widehat{\boldsymbol{\mu}}_x = \frac{1}{T}\sum_{t=1}^{T}\mathbf{x}_t\text{,} \qquad
  \widehat{\mathbf{\Sigma}}_x = \frac{1}{T-1}\sum_{t=1}^{T}\left(\mathbf{x}_t - \widehat{\boldsymbol{\mu}}_x\right)
  \left(\mathbf{x}_t - \widehat{\boldsymbol{\mu}}_x\right)^\intercal\text{.} 
\end{equation}

Ako pretpostavimo da slučajni vektor $\mathbf{X}$ dolazi is multivariatne normalne distribucije s
vektorom sredina $\boldsymbol{\mu}$ i kovarijaciskom matricom $\mathbf{\Sigma}$, tada je funkcija
izglednosti uzorka \engl{ likelihood function} dana jednadžbom \cite{statlect-likelihood}:
\begin{equation}
  \mathcal{L}\left(\boldsymbol{\mu}, \mathbf{\Sigma}; \mathbf{x_1}, \dots, \mathbf{x_T}\right) =
  \left(2\pi\right)^{-\frac{pT}{2}}|\det\left(\mathbf{\Sigma}\right)|^{-\frac{T}{2}}
  \exp\left(-\frac{1}{2}\sum_{t=1}^{T}\left(\mathbf{x}_t - \boldsymbol{\mu}\right)^\intercal
  \mathbf{\Sigma}^{-1}\left(\mathbf{x}_t - \boldsymbol{\mu}\right)\right)\text{.}
\end{equation}
S obzirom da taj oblik funckije izglednosti nije najprikladniji za računanje računalom iskazat ćemo i
njezin logaritamski oblik \engl{ log-likelihood function} te ćemo zamijeniti argument $\mathbf{\Sigma}$
s $\mathbf{\Sigma}^{-1}$:
\begin{equation} \label{eq:log-likelihood}
\begin{split}
  \ln\mathcal{L}&\left(\boldsymbol{\mu}, \mathbf{\Sigma}^{-1}; \mathbf{x_1}, \dots, \mathbf{x_T}\right) =\\
  &= \frac{1}{2}\left(
    -pT\ln(2\pi) + T\ln[\det(\mathbf{\Sigma}^{-1})] - \sum_{t=1}^{T}\left(\mathbf{x}_t - \boldsymbol{\mu}\right)^\intercal
    \mathbf{\Sigma}^{-1}\left(\mathbf{x}_t - \boldsymbol{\mu}\right)
  \right)\text{.}
\end{split}
\end{equation}


\section{Faktorski modeli}
\label{sec:faktorski}

U analizi povrata imovina često se primjenjuju multivarijatne statističke metode
kako bi se proučilo ponašanje i međusobna povezanost većeg broja vrijednosnica unutar portfelja.
Međutim, modeliranje velikog broja vremenskih nizova povrata dovodi do visokodimenzionalnih i
kompleksnih modela koji su složeni za interpretaciju i primjenu u praksi. Empirijska istraživanja
pokazuju da se povrati različitih vrijednosnica često kreću na sličan način, što nas upućuje da
postoje neki zajednički faktori koji utječu na njihovo kretanje. Primjerice, u razdobljima gospodarske
krize pad aktivnosti cijelog gospodarstva obično rezultira istodobnim padom cijena većine vrijednosnica,
dok rast određenog sektora, poput tehnološkog, često dovodi do rasta cijena većine dionica unutar
tog sektora \cite{anafin03}.

Te činjenice omogućuju faktorskim modelima da objasne kretanje većeg broja povrata pomoću
ogarničenog broja zajedničkih faktora te da pojednostavne njihovu analizu. Postoje tri vrste faktorskih
modela \cite{campbell1997}. Prvi su \emph{markoekonomski faktorski modeli} koji se fokusiraju na varijable
kao što su rast BDP-a, kamatne stope, stopa inflacije i slično. U ovakvom su modelu faktori osmotrivi
pa se model može estimirati linearnom regresijom. Drugi su \emph{fundamentalni faktorski modeli} koji
koriste podatke o poduzećima kako bi konsturiali svoje faktore. Treći su \emph{statistički faktorski
modeli} čiji su faktori neosmotrive latentne varijable koje se estimiraju iz podataka \cite{tsay2010}.
Ovaj rad će se fokusirati prvu vrstu odnosno na \emph{makroekonomske faktorske modele}.


\subsection{Linearni faktorski model}
\label{subsec:linearni-faktorski}

Pretpostavimo da imamo $p$ imovina kroz $T$ vremenskih trenutaka. Neka je $r_{it}$ povrat imovine $i$
u trenutku $t$. Opća forma faktorskog modela dana je izrazom:
\begin{equation}
  r_{it} = \alpha_i + \beta_{i1}f_{1t} + \dots + \beta_{im}f_{mt} + \epsilon_{it},
  \quad t=1,\dots,T; \quad i=1,\dots,p,
\label{eq:faktorski-opca}
\end{equation}
gdje je $\alpha_i$ konstanta, $\left\{f_{jt}|j=1,\dots,m\right\}$ su $m$ zajdeničkih (sistemskih)
faktora, $\beta_{ij}$ su koeficijenti imovine $i$ uz faktor $j$, a $\epsilon_{it}$ je specifičan
(idiosinkratski) faktor imovine $i$.

Za faktor $\mathbf{f}_t = \left(f_{1t}, \dots, f_{mt}\right)^\intercal$ vrijedi:
\begin{align}
  E(\mathbf{f}_t) &= \boldsymbol{\mu}_f, \\
  \mathrm{Cov}(\mathbf{f}_t) &= \mathbf{\Sigma}_f, \quad \text{$m \times m$ matrica,}
\end{align}
dok je idiosinkratski faktor $\epsilon_{it}$ modeliran bijelim šumom koji nije koreliran sa
sistemskim faktorima $f_{jt}$ i drugim idiosinkratskim faktorima. Dakle predpostavljamo:
\begin{align}
  E(\epsilon_{it}) &= 0 \quad \text{za sve } i,t \\
  \mathrm{Cov}(f_{jt}, \epsilon_{is}) &= 0 \quad \text{za sve } i,j,t,s \\
  \mathrm{Cov}(\epsilon_{it}, \epsilon_{js}) &=
  \begin{cases}
    \sigma_i^2, &\text{ako $i = j$ i $t = s$}, \\
    0, &\text{inače.}
  \end{cases}
\end{align}

Jednadžbu (\ref{eq:faktorski-opca}) možemo zapisati i u matričnom obliku za svih $p$ imovina
u trenutku $t$:
\begin{equation}
  \mathbf{r}_t = \boldsymbol{\alpha} + \boldsymbol{\beta}\mathbf{f_t} + \boldsymbol{\epsilon_t},
  \quad t = 1, \dots, T,
\label{eq:faktorski-matricna}
\end{equation}
gdje je $\mathbf{r}_t = \left(r_{1t}, \dots, r_{pt}\right)^\intercal$ vektor povrata,
$\boldsymbol{\alpha} = \left(\alpha_1, \dots, \alpha_p\right)^\intercal$ vektor konstanti,
$\boldsymbol{\beta} = \left[\beta_{ij}\right]$ je $p \times m$ matrica koeficijenata, a
$\boldsymbol{\epsilon}_t = \left(\epsilon_{1t}, \dots, \epsilon_{pt}\right)^\intercal$ je
vektor idiosinkratskih faktora čija je kovarijacijska matrica
$\mathrm{Cov}(\boldsymbol{\epsilon}_t) = \boldsymbol{\Psi} = \mathrm{diag}\left\{
\sigma_1^2, \dots, \sigma_k^2\right\} \; p \times p$ dijagonalna matrica. Kovarijacisku
matricu povrata $\mathbf{r}_t$ možemo računati sljedećom formulom:
\begin{equation}
  \mathrm{Cov}(\mathbf{r}_t) = \boldsymbol{\beta}\boldsymbol{\Sigma}_f\boldsymbol{\beta}^\intercal
  + \boldsymbol{\Psi}.
\end{equation}

Zatim jednadžbu (\ref{eq:faktorski-matricna}) možemo zapisati u sljedećem obliku:
\begin{equation}
  \mathbf{r}_t = \boldsymbol{\xi}\mathbf{g}_t + \boldsymbol{\epsilon}_t,
\end{equation}
gdje $\mathbf{g}_t = \left(1, \mathbf{f}_t^\intercal\right)^\intercal$,
a $\boldsymbol{\xi} = [\boldsymbol{\alpha}, \boldsymbol{\beta}]$ je
$p \times (m+1)$ matrica. Transponiramo li prethodnu jednadžbu i grupiramo podatke za svih $T$ trenutaka
dobivamo:
\begin{equation}
  \mathbf{R} = \mathbf{G}\boldsymbol{\xi}^\intercal + \mathbf{E},
\label{eq:faktorski-matricna-ukupna}
\end{equation}
gdje je $\mathbf{R} \; T \times p$ matrica povrata čiji je $t$-ti redak $\mathbf{r}_t^\intercal$,
$\mathbf{G}$ je $T \times (m+1)$ matrica čiji je $t$-ti redak $\mathbf{g}_t^\intercal$,
$\mathbf{E}$ je $T \times p$ matrica idiosinkratskih faktora čiji je $t$-ti
redak $\boldsymbol{\epsilon}_t^\intercal$ \cite{tsay2010}.

S obzirom da nas zanimaju \emph{makroekonomski faktorski modeli} čiji su faktori $\mathbf{f}_t$
osmotrivi, jednadžba (\ref{eq:faktorski-matricna-ukupna}) ima oblik višestruke multivariatne
linearne regresije. Zbog toga parametre modela možemo estimirati metodom najmanjih kvadrata
\cite{johnson2002}:
\begin{equation}
    \widehat{\boldsymbol{\xi}^\intercal} = \begin{bmatrix}
      \;\widehat{\boldsymbol{\alpha}}^\intercal\; \\
      \;\widehat{\boldsymbol{\beta}}^\intercal\;
    \end{bmatrix} = (\mathbf{G}^\intercal\mathbf{G})^{-1}(\mathbf{G}^\intercal\mathbf{R}),
\label{eq:ols-matrix}
\end{equation}
odakle su $\boldsymbol{\alpha}$ i $\boldsymbol{\beta}$ lako dostupni. Reziduale, odnosno
povrate idiosinkratskih faktora možemo lako dobiti koristeći formulu (\ref{eq:faktorski-matricna-ukupna}):
\begin{equation}
  \widehat{\mathbf{E}} = \mathbf{R} - \mathbf{G}\widehat{\boldsymbol{\xi}}^\intercal.
\end{equation}

\subsection{Jednofaktorski model povrata}
\label{subsec:jednofaktorski}

Jedan od najpoznatijih \emph{makroekonomskih faktorskih modela} koristi povrat tržišta
kao faktor koji utječe na sve vrijednosnice:
\begin{equation}
  r_{it} = \alpha_i + \beta_ir_{mt} + \epsilon_{it} \qquad i=1, \dots,p; \qquad t=1, \dots, T,
\label{eq:jednofaktorski}
\end{equation}
gdje je $r_{it}$ povrat vrijednosnice $i$ iznad bezrizične kamatne stope, a $r_{mt}$ povrat
tržišta iznad bezrizične kamatne stope. Kod modeliranja dionica, za povrat tržišta $r_{mt}$
uzima se povrat nekog tržišnog indeksa (npr. CROBEX za hrvatsko tržište) iznad bezrizične
kamatne stope. Koeficijenti modela $\alpha_i$ i $\beta_i$ procjenjuju se metodom najmanjih
kvadrata (\ref{eq:ols-matrix}).

Ovaj rad fokusirat će se upravo na jednofaktorski model povrata. Cilj ovog rada biti će
ispitati može li model dubokog učenja, iz prozora povijesnih povrata
$\mathbf{R}_H=\left\{\mathbf{r}_t\right\}_{t=1}^k$,
procijeniti koeficijente $\alpha_i$ i $\beta_i$, koji će bolje odgovarati budućem prozoru povrata
$\mathbf{R}_F=\left\{\mathbf{r}_t\right\}_{t=k+1}^T$,
nego procjena koeficijenata $\alpha_i$ i $\beta_i$ koju možemo dobiti metodom najmanjih kvadrata
na istom povijesnom prozoru povrata $\mathbf{R}_H$ za $1\le k < T$. Način na koji ćemo mjeriti
koliko dobro procjena koeficijenata $\alpha_i$ i $\beta_i$ odgovara budućem prozoru povrata
$\mathbf{R}_F$ biti će detaljnije objašnjen u potpoglavlju \ref{sec:model-dubokog-ucenja}.


\section{Duboko učenje}
\label{sec:duboko-ucenje}

Duboko učenje predstavlja podpodručje strojnog učenja koje se ističe u rješavanju problema s
visokom dimenzionalnošću podataka, kao što su računalni vid, obrada prirodnog jezika,
financijska te slične složene domene. Temeljna ideja dubokog učenja je izgradnja hijerarhijskih,
složenih reprezentacija podataka, koje se dobivaju primjenom uzastopnih nelinearnih transformacija
modeliranih pomoću višeslojnih neuronskih mreža. Osnovni element svake neuronske mreže je umjetni neuron.
Za zadani ulazni vektor $\mathbf{x} = (x_1, \dots, x_n)^\intercal$, izlaz neurona definiramo jednadžbom:
\begin{equation}
  h = f\left(\mathbf{w}^\intercal\mathbf{x} + b\right)
\end{equation}
gdje je $\mathbf{w} = (w_1, \dots, w_n)^\intercal$ vektor težina neurona koji odrežuje doprinos
pojedine komponente ulaznog vektora, a skalar $b$ omogućuje dodatni pomak linearne kombinacije ulaza.
Aktivacijska funkcija $f$ uvodi nelinearnost u model, čime se omogućuje aproskimacija složenih i
nelinearnih odnosa u podatcima.

Neke od najčešće korištenih aktivacijskih funkcija su sigmoidalna funkcija i hiperbolni tangens.
Obje su nelinearne, monotono rastuće i kontinuirano diferencijabilne, što je ključno svojstvo u
procesu učenja neuronskih mreža metodama koje se temelje na gradijentnom spustu. Sigmoidalna funkcija
ima kodomenu u intervalu $\left\langle0,1\right\rangle$, dok hiperbolni tangens poprima vrijednosti u
intervalu $\left\langle-1,1\right\rangle$, pri čemu obje funckije imaju karakterističan S-oblik.

Među najznačajnijim arhitekturama dubokih modela ističu se duboke unaprijende mreže, konvolucijske
neuronske mreže te povratne neuronske mreže. Svaka od navedenih arhitektura prilagođena je specifičnim 
vrstama podataka i problemima.

Odabir odgovarajuće arhitekture ovisi o prirodi i strukturi odabranih podataka. U okviru ovog rada
koristit ćemo posebnu vrstu povratnih neuronskih mreža koje nazivamo mreže s dugoročnom memorijom
\engl{ long short-term memory, LSTM}. LSTM mreže dizajnirane su kako bi omogućile učenje dugoročnih
ovisnosti u sekvencijalnim podatcima, zbog čega se često primjenjuju u modeliranju vremenskih nizova.

\subsection{Povratne neuronske mreže}
\label{subsec:povratne-mreze}

TODO Općenito objašnjenje povratnih neuronskih mreža


\subsection{LSTM mreže}
\label{subsec:lstm-mreze}

TODO Općenito objašnjenje LSTM mreža


%-------------------------------------------------------------------------------
\chapter{Implementacija}
\label{pog:implementacija}


\section{Podatci}
\label{sec:podatci}
TODO Objašnjenje odakle nam dolaze podatci, kako su procesirani te povezat s definiranim varijablama


\section{Model dubokog učenja}
\label{sec:model-dubokog-ucenja}

TODO Objasniti arhitekturu našeg modela
TOOD Objasniti različite funkcije cilja

%-------------------------------------------------------------------------------
\chapter{Rezultati}
\label{pog:rezultati}

TODO Rezultati i rasprava


%--- ZAKLJUČAK / CONCLUSION ----------------------------------------------------
\chapter{Zaključak}
\label{pog:zakljucak}

TODO Zaključak


%--- LITERATURA / REFERENCES ---------------------------------------------------

% Literatura se automatski generira iz zadane .bib datoteke / References are automatically generated from the supplied .bib file
% Upiši ime BibTeX datoteke bez .bib nastavka / Enter the name of the BibTeX file without .bib extension
\bibliography{literatura}



%--- SAŽETAK / ABSTRACT --------------------------------------------------------

% Sažetak na hrvatskom
\begin{sazetak}
  Unesite sažetak na hrvatskom.

\end{sazetak}

\begin{kljucnerijeci}
  prva ključna riječ; druga ključna riječ; treća ključna riječ
\end{kljucnerijeci}


% Abstract in English
\begin{abstract}
  Enter the abstract in English.

\end{abstract}

\begin{keywords}
  the first keyword; the second keyword; the third keyword
\end{keywords}


\end{document}
